\documentclass[UTF8]{ctexart}   %中文article文档类型排版,UTF8编码
\usepackage{amsmath}            %调用公式宏包
\usepackage{graphicx}           %插入图片宏包
\usepackage{hyperref}
\usepackage{ulem}               %插入下划线
\hypersetup{
            colorlinks,
            linkcolor=black,
            anchorcolor=blue,
            citecolor=green,
            CJKbookmarks=True
            }




\newtheorem{thm}{定理}               %定义标题为定理的定理类环境thm
\newcommand\degree{^\circ}           %定义新命令degree用来写角度的



\begin{document}



\title{第一讲 高数预备知识}
\date{\today}
\author{史学睿}
\maketitle

\tableofcontents


%第一节
\section{函数的概念与特性} 

1. 函数的定义

2. 反函数

\ \ 第一,\uuline{严格单调函数必有反函数},但有反函数,原函数不一定单调

第二,字母x和y互换才有,图像关于y=x对称

3. 复合函数


\end{document}