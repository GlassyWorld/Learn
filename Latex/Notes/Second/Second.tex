\documentclass[UTF8]{ctexart}       %中文article文档类型排版,UTF8编码
\usepackage{amsmath}                 %调用公式宏包
\usepackage{graphicx}                %插入图片宏包
\usepackage{hyperref}
\hypersetup{
            colorlinks,
            linkcolor=red,
            anchorcolor=blue,
            citecolor=green,
            CJKbookmarks=True
            }







\begin{document}



\title{Flexible PDMS-based triboelectric nanogenerator for instantaneous force sensing and human joint movement monitoring}
\date{\today}
\maketitle



%摘要
\begin{abstract}

    具有出色的电响应和自供电能力的柔性可穿戴传感器已成为个人医疗保健和人机界面
的一个新兴热点。

    在TENG基础上,一种柔性自供电式触觉传感器,由微截锥阵列结构的PDNS膜/铜电极
和聚偏二氟乙烯-三氟乙烯P(VDF-TrEE)纳米纤维已被证明,基于TENG的自供电触觉传感器,
可以在外部机械刺激下通过两个摩擦电层的接触分离过程产生电信号。

    采用微机电系统(MEMS)工艺制造的均匀且可控的微截锥阵列结构的P(VDF-TrEE)纳米纤维, 
基于柔性PDMS的传感器具有很高的灵敏度,具有2.97VkPa-1的稳定性,40000个循环的稳定性,
1Hz下60ms的响应时间,水滴的低检测压力(~4Pa,35mg)和低压区的良好线性关系。

    已有图案,叶子,丝状,微型金字塔,梯形,微截锥阵列结构

\end{abstract}

\tableofcontents %目录


%换页
\clearpage


%换页
\clearpage





% 参考文献
\addcontentsline{toc}{section}{参考文献} %用来添加文献的标准方式
\begin{thebibliography}{99}   %参考文献开始
    \bibitem{1} Yu J, Hou X, Cui M, et al. Flexible PDMS-based triboelectric nanogenerator for instantaneous force sensing and human joint movement monitoring[J]. Science China Materials, 2019, 62(10): 1423-1432.




\end{thebibliography}








\end{document}