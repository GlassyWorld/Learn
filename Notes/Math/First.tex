\documentclass[cn,normal,11pt,black]{elegantnote}
\usepackage{graphicx}
\usepackage{amsmath,amssymb}            %调用公式宏包
\usepackage{multirow}
\usepackage{lscape}
\usepackage{amsthm}



\title{高等数学预备知识}
\author{史学睿}
\date{\today}


\begin{document}
    \maketitle

    \section{函数的概念与特性}
    \subsection{函数}
    \subsection{反函数}
            
    设函数 $y=f(x)$ 的定义域为D,值域为R,如果对于每一个 $y \epsilon R$ ,必存在 $x \epsilon D$ ,使得
$y=f(x)$ 成立,则由此定义了一个新的函数 $x= \phi(y)$ .这个函数就成为函数 $y=f(x)$ 的\textbf{反函数},
 $x= f^{-1}(y)$ ,它的定义域为R,值域为D,相对于反函数来说,原来的函数也称为\textbf{直接函数},
以下两点需要说明:

第一,严格单调函数必有反函数,比如函数$y=x^2 (x \ \epsilon \ [0,+ \infty) )$是严格单调函数,
故它有反函数$x= \sqrt{y}$.

\underline{\textbf{有反函数的必定严格单调是不对的,要根据单调定义域判断}}

第二,若把 $x= f^{-1}(y)$ 与 $y=f(x)$ 的图形画在同一坐标系中,则它们完全重合.只有把 $y=f(x)$ 的反函数$x= f^{-1}(y)$
写成$y=f^{-1}(x)$之后,它们的图形才关于$y=x$对称,\underline{事实上这也是字母 $x$ 与 $y$ 互换的结果}。

第三,求解反函数的一般步骤为:
\begin{itemize}
    \item 求解原函数的单调区间及值域
    \item 求解 $ x $ ,然后互换 $x$ 和 $y$
\end{itemize}

    \subsection{复合函数}

    双曲正弦函数 

    \begin{equation}
        sinhx = \frac{e^x-e^{-x}}{2}
    \end{equation}

    \clearpage

    反双曲正弦函数 

    \begin{equation}
        y = ln[x+\sqrt{x^2 + 1}] 
    \end{equation}
    两个求导基本公式
    \begin{align*}
        \int \frac{1}{\sqrt{x^2 + 1}}dx & = ln[x+ \sqrt{x^2+1}] \\
        \\
        (ln[x+\sqrt{x^2+1}])^{'} &= \frac{1}{\sqrt{x^2+1}} \\
    \end{align*}

    复合函数最重要的就是两个函数定义域与值域的区分,尤其注意:
    
    \underline{\textbf{题目给出的函数定义域是否完全是正确的定义域}}

    \subsection{四种特性}
    \subsubsection{有界性}
        设$ f(x) $ 的定义域为 $ D  $ ,数集 $ I \Subset D $,如果存在某个正数$ M $ ,使得
    对任一$ x \in I $ ,有 $ |f(X)| \leq m $ ,则称 $ f(x) $ 在$I$ 上有界;如果这样的$M$ 不存在,
    则称$f(x)$ 在 $ I $ 上无界。
        \begin{note}
            \begin{itemize}
            \item 有界还是无界的讨论,首先要\textbf{指明区间},不知区间,无论有界性
            \item 事实上,只要区间$I$ 上存在点 $x_0$ ,使得函数$\lim\limits_{x \to x_0}f(x)$的值
            为无穷大,这就叫无界
            \end{itemize}
        \end{note}
    \subsubsection{单调性}
    设$f(x)$ 是定义在$ [-l,l] $上的任意函数,则 \\
    $ F_1(x)=f(x)-f(-x) $必为\textbf{奇函数}; $ F_2(x)=f(x)+f(-x) $必为\textbf{偶函数}
    \begin{itemize}
        \item 奇函数$y=f(x)$ 的图形关于坐标原点对称,当$f(x)$ 在$x=0$ 处有定义时,必有$f(0)=0$
        \item 偶函数$y=f(x)$ 的图形源于$y$轴对称,且当$f^{'}(0)$ 存在时必有$f^{'}(0)=0$
        \item 函数$y=f(x)$ 的图形关于直线$x=T$对称的充分必要条件是
        $$ f(x)=f(2T-x) \text{或} f(T+x)=f(T-x) $$
    \end{itemize}
    \subsubsection{奇偶性}
    \subsubsection{周期性}


    


\end{document}