\documentclass[UTF8]{ctexart}   %中文article文档类型排版,UTF8编码
\usepackage{amsmath,amssymb}            %调用公式宏包
\usepackage{hyperref}
\hypersetup{
            colorlinks,
            linkcolor=red,
            anchorcolor=blue,
            citecolor=green,
            CJKbookmarks=True
            }
%图片设置
\usepackage{graphicx}           %插入图片宏包
\usepackage{caption2}
\renewcommand{\captionlabeldelim}{.}
\usepackage{import}             %使用相对于子文件的路径
\usepackage{pdfpages}           %引用pdf
\usepackage{xifthen}
\usepackage{transparent}
\usepackage{textcomp}           %特殊符号

\begin{document}



\title{关于Springer物理相关书籍的简介及目录}
\date{\today}
\maketitle


注意:该PDF中的蓝色部分为链接到本地文件的超链接,在WPS中可以使用(因为我只试过
WPS),若无法打开请自行去文件夹下寻找。

Springer简介:

Springer即施普林格,是世界著名的科技期刊、图书出版公司,是全球第一大科技图书出
版公司和第二大科技期刊出版公司。1842年在德国柏林创立,每年出版6500余种科技图书
和约2000余种科技期刊。

疫情期间,为了支持高等教育机构的教学工作,Springer Nature免费提供500余本关键
教科书的访问权限,其中英文教科书409种,教科书涵盖诸多学科,如行为科学及心理学、
生物医学及生命科学、商业、经济、管理、金融、化学及材料科学、计算机科学、地球
与环境科学、教育学、能源、工程、智能技术及机器人、数学与统计科学、医学、物理及
天文学、社会科学等,免费访问/下载全文至2020年7月31日。免费教材通过\href{https://link.springer.com/}{SpringerLink}访问,详细清单及
部分图书介绍在\href{run:./Physics/Books.xlsx}{书单}中列出。

为方便查阅,我自行整理了物理学相关的一些教材放在该压缩包中,大家点击蓝色字体即可
跳转(不能跳转的请看开头)。由于原教材均为英文,这里我Google机翻+个人理解简单
翻译了下书名。

\clearpage

\centerline{\songti \Large \textbf{量子力学部分}}

\href{run:./Physics/A First Introduction to Quantum Physics.pdf\ }{量子力学初学介绍}

\href{run:./Physics/Foundations of Quantum Mechanics_.pdf\ }{量子力学基础}

\href{run:./Physics/Introductory Quantum Mechanics.pdf\ }{量子力学概论}

\href{run:./Physics/Principles of Quantum Mechanics.pdf\ }{量子力学原理}

\href{run:./Physics/Quantum Mechanics Short Version.pdf\ }{量子力学简版}

\href{run:./Physics/Quantum Mechanics Full Version.pdf\ }{量子力学全版}

\href{run:./Physics/Advanced Quantum Mechanics.pdf\ }{先进量子力学}

\href{run:./Physics/Quantum Mechanics for Pedestrians 1.pdf\ }{给路人看的量子力学1}

\href{run:./Physics/Quantum Mechanics for Pedestrians 2.pdf\ }{给路人看的量子力学2}

\href{run:.//Phsics/Quantum Theory for Mathematicians_.pdf\ }{数学家的量子理论}

\centerline{\songti \Large \textbf{其他}}

\href{run:./Physics/Solid-State Physics.pdf\ }{固体物理学}

\href{run:./Physics/The Physics of Semiconductors.pdf\ }{半导体物理学}

\href{run:./Physics/Mechanics And Thermodynamics.pdf\ }{力学与热学}

\href{run:./Physics/Basics of Laser Physics.pdf\ }{激光物理基础}

\href{run:./Physics/Problems in Classical Electromagnetism.pdf\ }{经典电磁学中的问题}

\href{run:./Physics/Introduction to General Relativity.pdf\ }{广义相对论导论}

\href{run:./Physics/Mathematical Physics.pdf\ }{数学物理}

\href{run:./Physics/Partial Differential Equations.pdf\ }{偏微分方程}

\href{run:./Physics/Applied Partial Differential Equations_.pdf\ }{应用偏微分方程}

\href{run:./Physics/Probability Theory.pdf\ }{概率论}

\href{run:./Physics/Basic Concepts in Computational Physics_.pdf\ }{计算物理的基本概念}

\href{run:./Physics/Computational Physics_.pdf\ }{计算物理}

\href{run:./Physics/Design and Analysis of Experiments_.pdf\ }{实验设计与分析}

\href{run:./Physics/Linear Algebra and Analytic Geometry for Physical Sciences.pdf\ }{物理学中的线代和解几}

\href{run:./Physics/Physics of Oscillations and Waves.pdf\ }{震荡与波的物理学}

\href{run:./Physics/Structure Determination by X-ray Crystallography_.pdf\ }{X射线晶体学测定结构}

\clearpage

以下两本是我个人推荐大家学习一下的书籍。《Knowledge Management》知识管理,懂的都懂,
不需要我多说了。至于LaTeX,这是一个基于TeX的排版系统,利用这种格式,可以方便的
排版高质量的书籍,特别是包含\underline{\textbf{数学公式}}的书籍,并因数学排版能力的出色,而被
学术界广泛使用。虽然入门不像Word那样简单,但是只需要经过简单的学习,就可以自己整理
出一个适合自己的模板框架,之后就是不断重复利用和修补这个框架了。对我而言,LaTeX最棒的
地方就是你完全不需要担心排版,也无需鼠标点来点去的进行繁琐的公式输入工作,搭配mathpix
软件甚至只需要复制粘贴,就可以得到一个个精美的公式。当然,Word+mathpix+mathtype也可以
达到较为方便的输入公式的效果,但是排版方面,比如图片的插入与排布,公式的编号等等一系列
、神奇且记不住的操作了。这个.tex文件我会保留,感兴趣的话,大家可以看一下。另外的教材
由于总体积太大,我会上传到百度网盘,上传完成后,我会在群里发一个链接。

\href{run:./Physics/Knowledge Management.pdf\ }{知识管理}

\href{run:./Physics/LaTeX in 24 Hours.pdf\ }{24小时学会LaTeX}



\end{document}